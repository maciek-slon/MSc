% !TeX root = main.tex


\begin{savequote}[80mm]
\begin{enumerate}
\item Robot nie może skrzywdzić człowieka, ani przez zaniechanie działania dopuścić, aby człowiek doznał krzywdy.\\
\item Robot musi być posłuszny rozkazom człowieka, chyba że stoją one w~sprzeczności z~Pierwszym Prawem.\\
\item Robot musi chronić sam siebie, jeśli tylko nie stoi to w~sprzeczności z~Pierwszym lub Drugim Prawem.\end{enumerate}
\qauthor{Isaac Asimov}
\end{savequote}


\chapter{Wstęp}
\label{chap:wstep}

\section{Motywacja}

Nawigacja autonomicznego bądź semiautonomicznego robota mobilnego wymaga możliwie
efektywnych i dokładnych metod analizy otoczenia oraz wykrywania przeszkód.
Najczęściej stosuje się w tym celu całe zestawy różnych czujników, z których
podaje odczyty bądź punktowe (np. czujniki zderzeniowe bądź sensory odległości
wykorzystujące podczerwień albo ultradźwięki) bądź czasami dwuwymiarowe (skanery
laserowe podające odległość do otaczających przedmiotów, jednak odczyty wykonywane
jedynie w jednej płaszczyźnie). Taka konfiguracja czujników umożliwia dość sprawne
poruszanie się w nieznym środowisku, jednak istnieje szereg przesgód, których za
ich pomocą nie da się wykryć. Należą do nich chociażby małe obiekty znajdujące
się bardzo blisko ziemi (czujniki najczęściej są umieszczone na wysokości od kilku
do kilkunastu centymetrów) czy obiekty zwisające z góry mogące zahaczyć o wystające
części przejeżdżającego pod nimi robota.

Wykorzystanie informacji dostarczanych przez czujniki 3D (a więc zwracające informacje
o głębi w każdym punkcie obserwowanego obrazu w przypadku czujników opartych o kamery)
pozwala na wykrywanie dużo szerszego spektrum obiektów, a więc sprawniejsze i bezpeczniejsze
poruszanie się do okreslonego celu.

\section{Cel pracy}

Celem pracy jest stworzenie pełnego systemu sterowania robota mobilnego,
wykorzystującego informacje o otoczeniu z wielu źródeł, w tym z kamery 3D.

\section{Struktura pracy}


