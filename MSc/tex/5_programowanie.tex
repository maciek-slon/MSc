% !TeX root = main.tex


\begin{savequote}[70mm]
,,''
\qauthor{}
\end{savequote}


\chapter{Programowanie robotów mobilnych}
\label{chap:programowanie}

\section{Metody programowania robotów}

Do wykorzystania wszystkich możliwości dawanych przez sprzęt zainstalowany na robocie
wymagane jest odpowiednie oprogramowanie na komputerze sterującym robota. Odpowiada
ono za zadania na wielu poziomach abstrakcji, od sterowników niskiego poziomu
komunikujących się bezpośrednio ze sprzętem, przez moduły przetwarzające dane otrzymane
bezpośrednio z sensorów do pewnej, z góry założonej postaci, aż po wysokopoziomowe
algorytmy wykorzystujące przygotowane wcześniej dane do wykonania pewnego zadania.

\section{Sterowniki wbudowane}

\subsection{Ograniczenia}

\section{Ramowe struktury programowe}


W ciągu ostatnich lat powstało wiele ramowych struktur programowych do sterowania
robotami, niektóre z nich specjalizowane dla jednego typu robotów (np. manipulatorów),
inne przygotowane do sterowania całą gamą różnych robotów, od platform mobilnych,
przez różnego rodzaju manipulatory, na robotach latających i pływających kończąc.
W tej części rozdziału przedstawię porównanie kilku struktur przeznaczonych do
sterowania m.in. robotami mobilnymi, wraz z zestawieniem wad i zalet każdego z rozwiązań
oraz uzasadnieniem wyboru jednego z nich.


\subsection{Player/Stage}

\cite{gerkey03playerstage}

\subsection{ROS}

\cite{288}
