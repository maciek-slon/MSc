% !TeX root = main.tex


\begin{savequote}[70mm]
,,Słaby wątpi przed decyzją, silny po niej.''
\qauthor{Karl Kraus}
\end{savequote}


\chapter{Programowanie robotów mobilnych}
\label{chap:programowanie}

\section{Metody programowania robotów}

Do wykorzystania wszystkich możliwości dawanych przez sprzęt zainstalowany na robocie
wymagane jest odpowiednie oprogramowanie na komputerze sterującym robota. Odpowiada
ono za zadania na wielu poziomach abstrakcji, od sterowników niskiego poziomu
komunikujących się bezpośrednio ze sprzętem, przez moduły przetwarzające dane otrzymane
bezpośrednio z~sensorów do pewnej, z~góry założonej postaci, aż po wysokopoziomowe
algorytmy wykorzystujące przygotowane wcześniej dane do wykonania pewnego zadania.

\section{Sterowniki wbudowane}

W przypadku prostych robotów, wyposażonych w~niewielką ilość czujników
i~przeznaczonych do wykonywania kilku z~góry określonych zadań stosuje się systemy
złożone z~mikrokontrolera połączonego z~układami sensorycznymi i~sterującego pracą
całego robota. Do jego zadań należy zarówno odczyt danych z~czujników, ich analiza
i~przetwarzanie, a~także sterowanie efektorami robota. Systemy takie mają niewielką
wydajność, ale ze względu na ich dobre dostosowanie do przewidzianego zadania działają
szybko i~pozwalają na uruchomienie pętli sterowania nawet w~tempie kilkuset obiegów na sekundę.
Przykładem takiego systemu może być układ sterowania robotów sportowych. W~przypadku
robotów śledzących linię ich prędkość często przekracza 2m/s, przy jednoczesnej
akwizycji i~analizie danych z~kilkunastu czujników i~zdolności do natychmiastowej
reakcji na wykryte przeszkody.

Największą zaletą takich systemów jest ich niski koszt (często oscylujący w~okolicach
kilkunastu PLN) i~prostota konstrukcji (możliwość montażu na płytce uniwersalnej).
Niestety, możliwości są mocno ograniczone, a~zmiana zadania wymaga często oprócz przepisania
samego algorytmu sterującego także zmian sprzętowych. Rozwiązanie tego typu nie nadaje
się do przygotowania platformy badawczej.

\section{Programowe struktury ramowe}

Drugą metodą jest podział systemu sterowania na rozłączne elementy -- samego
robota (efektor) i~komunikujący się z~nim komputer nadrzędny (który może być umieszczony
zarówno na robocie, jak i~poza nim sterując nim zdalnie). W~tym przypadku podzespoły
robota (jego układy sensoryczne, silniki itp.) mają swoje odrębne sterowniki sprzętowe
(np. układ sterownika silników może przyjmować jako sterowanie zadaną prędkość
i~na jej podstawie generować odpowiednie sygnały dla silników, układy sensoryczne
mogą agregować dane z~wielu czujników i~wysyłać je we wstępnie przetworzonej formie
do komputera sterującego). Na komputerze sterującym uruchomione jest oprogramowanie
komunikujące się ze sterownikami sprzętowymi i~udostępniające pozyskane w~ten sposób
dane zewnętrznym algorytmom sterowania.

Metody takie są elastyczne, umożliwiają łatwą zmianę zadania, jakie ma wykonywać robot,
a~także wymianę poszczególnych fragmentów systemu sterowania. Dodatkowo same aplikacje
działają niejako niezależnie od sprzętu na robocie, więc możliwe jest wykorzystanie
raz napisanych algorytmów na innych robotach (o ile dostarczają one wymaganych danych
pomiarowych), jak też łatwo można zastąpić rzeczywistego robota jego symulacją w~celach
testowych. Problemem jest dużo większa komplikacja całego układu -- wymagany jest
komputer sterujący o~dużej mocy (aby możliwe było wykonywanie różnych zadań i~obsługa
komunikacji), osobne sterowniki sprzętowe dla podzespołów robota (nie da się sterować
wszystkimi elementami bezpośrednio) oraz niezawodnie działająca struktura programowa
rozdzielająca algorytmy sterowania od warstwy fizycznej.

W~ciągu ostatnich lat powstało wiele ramowych struktur programowych do sterowania
robotami, niektóre z~nich specjalizowane dla jednego typu robotów (np. manipulatorów),
inne przygotowane do sterowania całą gamą różnych robotów, od platform mobilnych,
przez różnego rodzaju manipulatory, na robotach latających i~pływających kończąc.
W~tej części rozdziału przedstawiono porównanie kilku struktur przeznaczonych do
sterowania m.in. robotami mobilnymi, wraz z~zestawieniem wad i~zalet każdego z~rozwiązań
oraz uzasadnieniem wyboru jednego z~nich.


\subsection{Player/Stage}

Pakiet oprogramowania Player/Stage\footnote{\footurl{playerstage.sourceforge.net}}
składa się z~trzech elementów: serwera Player (pośredniczącego w~wymianie danych
pomiędzy robotem a~układem sterowania), oraz wielorobotowych symulatorów Stage (2D)
i~Gazebo (3D). System zbudowany jest w~architekturze klient-serwer, gdzie serwer
uruchomiony jest na komputerze, który bezpośrednio steruje robotem i~odbiera
pomiary z~czujników, a~programy klientów wykorzystując dostarczone interfejsy
pobierają dane sensoryczne i~odsyłają komendy sterujące.~\cite{gerkey03playerstage}
Zadaniem serwera jest abstrakcja aplikacji klienckich od kwestii sprzętowych.
Komunikacja pomiędzy serwerem i~klientami (może ich być podłączonych wielu)
odbywa się poprzez systemowe gniazda sieciowe TCP, dzięki czemu możliwe jest także
zdalne sterowanie robotem przez programy klienckie uruchomione na komputerach poza
robotem i~komunikujące się z~nim przez sieci bezprzewodowe.

System ten został stworzony specjalnie do sterowania robotami mobilnymi oraz sieciami
czujników, od czasu jego powstania zostało w~nim zaimplementowane wiele sterowników
do popularnych robotów (np. Pioneer2DX) oraz sensorów (np. Sick LMS200). Istnieje
też pokaźna baza algorytmów przygotowanych do zastosowania na robotach uruchamianych
z~systemem Player. Największą jego wadą jest narzucony sposób komunikacji, czyli
jedynie pomiędzy serwerem a~klientami (możliwe jest uzyskanie komunikacji klient-klient
oraz serwer-serwer, nie są to jednak rozwiązania standardowe i~łatwe w~implementacji).
Z tego powodu utrudnione jest budowanie wielowarstwowych systemów nawigacji z~wieloetapowym
przetwarzaniem danych sensorycznych -- całe przetwarzanie musi odbywać się w~aplikacji
serwera (wszystkie sterowniki uruchamiane są jako osobne wątki), co może negatywnie
odbić się na wydajności.

\subsection{ROS}

System ROS\footnote{Robot Operating System\footurl{www.ros.org}} stworzony został
do sterowania złożonymi platformami do mobilnej manipulacji, wyposażonymi w~wiele
czujników oraz efektorów, w~szczególności manipulatory. Jest bardziej złożony od
opisanego wcześniej systemu Player/Stage, kładąc dużo większy nacisk na rozproszenie
obliczeń na wiele maszyn. Jest to pewnego rodzaju meta-system o~architekturze rozproszonej,
zawierający warstwę abstrahującą algorytmy od sprzętu, mechanizmy komunikacji
pomiędzy różnymi procesami, zaawansowane zarządzanie pakietami modułów oraz zestaw
aplikacji do zarządzania ich wykonaniem na wielu maszynach~\cite{288}.

W przeciwieństwie do aplikacji w~systemie Player, tutaj każdy uruchomiony proces
może pełnić rolę zarówno serwera jak i~klienta (a także obu jednocześnie), umożliwiając
pełne wykorzystanie mocy obliczeniowej dostępnych w~sieci komputerów do przetwarzania
danych (każdy proces pobiera dane od swojego poprzednika, dokonuje kolejnego przekształcenia,
a~wyniki od niego odbierane są przez jego następnik, każdy z~procesów może działać na
innej maszynie). Należy zaznaczyć, że w~systemie ROS możliwe jest uruchomienie serwera Player jako
jeden z~procesów, oraz tworzenie węzłów korzystających z~danych udostępnianych przez
niego. W~ten sposób umożliwiono łatwą migrację do nowego systemu (rozwijanego częściowo
przez ten sam zespół programistów co Player). System ROS wykorzystuje również symulator
Gazebo.

System ROS został wybrany do wykonania zadań tej pracy ze względu na swoją nowoczesną
architekturę, mnogość gotowych algorytmów nawigacji, a~także dostępne sterowniki
do sensora Kinect (mające oficjalne wsparcie autorów tej technologii). Ze względu
na możliwość rozproszenia przetwarzania na wiele maszyn możliwe stało się odciążenie
głównego komputera sterującego robotem (którego wydajność nie jest wysoka) i~przeniesienie
części obliczeń na maszyny zewnętrzne (np. globalnego planowania trasy, tworzenia
mapy itp).
