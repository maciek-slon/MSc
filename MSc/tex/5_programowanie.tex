% !TeX root = main.tex


\begin{savequote}[70mm]
,,''
\qauthor{}
\end{savequote}


\chapter{Programowanie robotów mobilnych}
\label{chap:programowanie}

\section{Metody programowania robotów}

Do wykorzystania wszystkich możliwości dawanych przez sprzęt zainstalowany na robocie
wymagane jest odpowiednie oprogramowanie na komputerze sterującym robota. Odpowiada
ono za zadania na wielu poziomach abstrakcji, od sterowników niskiego poziomu
komunikujących się bezpośrednio ze sprzętem, przez moduły przetwarzające dane otrzymane
bezpośrednio z sensorów do pewnej, z góry założonej postaci, aż po wysokopoziomowe
algorytmy wykorzystujące przygotowane wcześniej dane do wykonania pewnego zadania.

\section{Sterowniki wbudowane}

W przypadku prostych robotów wyposażonych w niewielką ilość czujników
i przeznaczonych do wykonywania kilku z góry okreslonych zadań stosuje się systemy
złożone z mikrokontrolera połączonego z układami sensorycznymi i sterującego pracą
całego robota. Do jego zadań należy zarówno odczyt danych z czujników, ich analiza
i przetwarzanie, a także sterowanie efektorami robota. Systemy takie mają niewielką
wydajność, ale ze względu na ich dobre dostosowanie do przewidzianego zadania działają
szybko i pozwalają na uruchomienie pętli sterowania nawet w tempie kilkuset na sekundę.
Przykładem takiego systemu może byc układ sterowania robotów sportowych. W przypadku
robotów śledzących linię ich prędkość często przekracza 2m/s, przy jednoczesnej
akwizycji i analizie danych z kilkunastu czujników i zdolności do natychmiastowej
reakcji na wykryte przeszkody.

Największą zaletą takich systemów jest ich niski koszt (często oscylujący w okolicach
kilkunastu PLN) i prostota konstrukcji (możliwość montażu na płytce uniwersalnej).
Niestety, możliwości są mocno ograniczone, a zmiana zadania wymaga często oprócz przepisania
samego algorytmu sterującego także zmian sprzętowych. Rozwiązanie tego typu nie nadaje
się do przygotowania platformy badawczej.

\section{Ramowe struktury programowe}

Drugą metodą jest podział systemu sterowania na rozłączne elementy -- samego
robota (efektor) i komunikujący się z nim komputer nadrzędny (który może być umieszczony
zarówno na robocie, jak i poza nim sterując nim zdalnie). W tym przypadku podzespoły
robota (jego układy sensoryczne, silniki itp.) mają swoje odrębne sterowniki sprzętowe
(np. układ sterownika silników może przyjmować jako sterowanie zadaną prędkość
i na jej podstawie generować odpowiednie sygnały dla silników, układy sensoryczne
mogą agregować dane z wielu czujników i wysyłać je we wstępnie przetworzonej formie
do komputera sterującego). Na komputerze sterującym uruchomione jest oprogramowanie
komunikujące się ze sterownikami sprzętowymi i udostępniające pozyskane w ten sposób
dane zewnętrznym algorytmom sterowania.

Metody takie są elastyczne, umożliwiają łatwą zmianę zadania, jakie ma wykonywac robot,
a także wymianę poszczególnych fragmentów systemu sterowania. Dodatkowo same aplikacje
działają niejako niezależnie od sprzętu na robocie, więc możliwe jest wykorzystanie
raz napisanych algorytmów na innych robotach (o ile dostarczają one wymaganych danych
pomiarowych), jak też łatwo można zastąpić rzeczywistego robota jego symulacją w celach
testowych. Problemem jest dużo wieksza komplikacja całego układu -- wymagany jest
komputer sterujący o dużej mocy (aby możliwe było wykonywanie różnych zadań i obsługa
komunikacji), osobne sterowniki sprzętowe dla podzespołów robota (nie da się sterować
wszystkimi elementami bezpośrednio) oraz niezawodnie działająca struktura programowa
abstrachująca algorytmy sterowania od warstwy fizycznej.

W ciągu ostatnich lat powstało wiele ramowych struktur programowych do sterowania
robotami, niektóre z nich specjalizowane dla jednego typu robotów (np. manipulatorów),
inne przygotowane do sterowania całą gamą różnych robotów, od platform mobilnych,
przez różnego rodzaju manipulatory, na robotach latających i pływających kończąc.
W tej części rozdziału przedstawię porównanie kilku struktur przeznaczonych do
sterowania m.in. robotami mobilnymi, wraz z zestawieniem wad i zalet każdego z rozwiązań
oraz uzasadnieniem wyboru jednego z nich.


\subsection{Player/Stage}

\cite{gerkey03playerstage}

\subsection{ROS}

\cite{288}
