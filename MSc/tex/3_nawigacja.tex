% !TeX root = main.tex


\begin{savequote}[70mm]
,,''
\qauthor{}
\end{savequote}

\chapter{Nawigacja robotów mobilnych}
\label{chap:nawigacja}

Całe zadanie nawigacji robota mobilnego może zostać podzielone na kilka głównych
składników \cite[cz.~9]{szynkiewiczWR}:

\begin{itemize}
  \item precepcję,
  \item samolokalizację robota,
  \item wnioskowanie i/lub planowanie,
  \item tworzenie mapy otoczenia,
  \item generowanie trajektorii i omijanie przeszkód.
\end{itemize}

W zależności od potrzeb, nie wszystkie z nich muszą być uwzględnione w systemie
sterowania (np. roboty mogą poruszać się w terenie, którego mapa jest nieznana
i nie jest wymagane jej tworzenie, bądź też a priori zakłada się brak
przeszkód na trasie i nie uwzględnia problemu ich wykrywania). Niniejszy
rozdział opisuje ogólne metody realizacji poszczególnych elementów systemu.

\section{Percepcja}

Ta część systemu odpowiada za gromadzenie wiedzy o stanie zarówno samego robota,
jak i otaczającego go środowiska, w którym operuje. Dokonuje się tego przez
akwizycję danych z czujników a następnie wybieranie tych informacji, które mają
znaczenie w nawigacji. 

\subsection{Klasyfikacja czujników}

Istnieje cały szereg różnych
czujników, które ogólnie można pogrupować ze względu na rodzaj mierzonych
wielkości (wyróżniamy proprioreceptory -- czujniki mierzące wartości parametrów
wewnętrznych robota,oraz eksteroreceptory -- mierzące wartości parametrów
otoczenia) oraz wpływ na otoczenie (tutaj mamy podział na czujniki pasywne i
aktywne)~\cite{siegwart}.


\subsubsection{Proprioreceptory i eksteroreceptory}

Proprioreceptory mierzą parametry określające w pewien sposób wewnętrzny stan
robota, takie jak prędkość silników czy napięcie akumulatorów. Wykorzystywane są
przy wyznaczaniu pozycji robota (czujniki odometryczne), sterowaniu (prędkość
obrotowa silników) czy diagnostyce jego podzespołów. Eksteroreceptory 
pozyskują informację z otoczenia robota, takie jak natężenie
światła czy amplituda dźwięku. Można za ich pomocą mierzyć odległość,
orientację, lokalizować zewnętrzne obiekty aż wreszcie budować mapy.

Ważne jest to, że konkretne sensory są przypisywane do proprio- lub
eksteroreceptorów w zależności od ich zastosowania, a nie typu mierzonej
wartości (np. czujnik temperatury zamontowany na radiatorze silnika jest
czujnikiem wewnętrznym, a ten sam czujnik mierzący temperaturę otoczenia robota
jest już czujnikiem zewnętrznym).

\subsubsection{Pasywne i aktywne}

Czujniki pasywne rejestrują jeydnie energię przychodzącą z otoczenia i na
jej podstawie określają wielkość mierzonego parametru. Do takich czujników 
należą termometry, mikrofony, kamery wizyjne, czy czujniki uderzeniowe.
Czujniki aktywne do wykonania pomiaru wymagają wyemitowania dodatkowej energii
do otoczenia, a następnie pomiar wykonywany jest na podstawie pomiaru ilości
energii, która powróciła do sensora. Zaletą czujników pasywnych jest brak
dodatkowych zakłóceń wprowadzanych do otoczenia, które mogą się pojawić 
przy korzystaniu z czujników aktywnych (które mogą się pojawić przy korzystaniu
z wielu takich samych czujników na jednym bądź wielu robotach), problemem 
przy czujnikach aktywnych jest także możliwość zafałszowania samej mierzonej
wartości przez emitowaną energię (np. powstawanie odblasków). Największą zaletą
czujników aktywnych jest możliwość pracy w trudniejszych warunkach (np. nie
jest wymagane oświetlenie zewnętrzne przy pomiarach odległości).

\subsection{Podstawowe klasy czujników}

Czujniki dostępne w robotyce można wreszcie podzielić na klasy ze względu 
na metodę ich działania. poniżej przedstawione są główne klasy, w porządku
rosnącym pod względem skomplikowania konstrukcji, a malejącym ze względu na
dojrzałość konstrukcji~\cite{siegwart}.

\subsubsection{Czujniki dotykowe}
\subsubsection{Czujniki odometryczne}
\subsubsection{Czujniki orientacji}
\subsubsection{Naziemne latarnie kierunkowe}
\subsubsection{Aktywne czujniki odległości}
\subsubsection{Czujniki prędkości i ruchu}
\subsubsection{Czujniki wizyjne}

\section{Samolokalizacja}

Pod pojęciem samolokalizacji robota rozumie się określenie bieżącej pozycji
robota w pewnym, ustalonym układzie odniesienia. Do obliczeń wykorzystywane mogą
być dane uzyskiwane z czujników robota, 

\subsection{Klasyfikacja metod lokalizacji}

Metody lokalizacji robotów mobilnych można podzielić biorąc pod uwagę różne
kryteria. Kilka możliwych podziałów przedstawione jest w dalszej części
rozdziału, wraz z przykładami.

\subsubsection{Metody lokalne i globalne}

W zależności od lokalizacji układu odniesienia metody lokalizacji możemy
podzielić na lokalne i globalne. W metodach lokalnych układ odniesienia, w
którym następuje lokalizacja jest związany z punktem, z którego robot rozpoczyna
pracę (a więc początkowe położenie robota jest zerowe bądź wskazane przez
operatora), tak więc rozpoczynając pracę z faktycznie różnych punktów i
wykonując tę samą trajektorię (zakładając, że w ogóle może ją w każdym
przypadku wykonać), mimo takiej samej pozycji wyliczonej przez układ
lokalizacji robot będzie się znajdował w różnych miejscach.

Metody globalne operują z kolei w układach odniesienia związanych z elementami,
które nie są bezpośrednio zależne od robota. Mogą to być układy związane z danym
pomieszczeniem (z ustalonym odgórnie punktem zerowym) bądź takie jak układ
współrzędnych geograficznych, jak również oparte o mapę i lokalizujące robota
względem umieszczonych na niej obiektów. W takich metodach kluczowe jest
określenie pozycji początkowej, które nie zawsze jest łatwe do wykonania.
Korzystając z odbiorników GPS łatwo zlokalizować robota w przestrzeni otwartej,
trudniej jest to wykonać w pomieszczeniach. Rozwiązują to metody czynne, których
opis znajduje się w dalszej części rozdziału.

\subsubsection{Środowisko statyczne i  dynamiczne}

Poprzez określenie środowiska statycznego rozumie się takie otoczenie, w którym
ruch innych obiektów oprócz robota nie występuje bądź jest niewykrywalny przez
jego czujniki. Podczas pracy w środowisku dynamicznym (a więc takim, gdzie
pomimo braku ruchu robota czujniki rejestrują ruch przeszkód) algorytmy
lokalizacji muszą uwzględniać ten fakt i stosować różne rodzaje filtrowania
danych tak, aby ruch przeszkód wokół robota nie powodował zmian jego wyliczonego
położenia. Metodami odpornymi na zmiany w otoczeniu robota są wszystkie te,
które wykorzystują informacje o przesunięciu odczytywane z czujników
przesunięcia (np. odometryczne bądź inercyjne), wrażliwe mogą być z kolei metody
wykorzystujące odczyty laserowe.

\subsubsection{Metody bierne i czynne}

Przy biernych metodach lokalizacji do określenia pozycji robota jego ruch
(a dokładniej ruch sensorów) nie jest konieczny. Tak działa chociażby GPS, oraz
wszystkie metody wykorzystujące zewnętrzne nadajniki (np. latarnie radiowe). W
metodach czynnych do dokładnego określenia pozycji robota wymagany jest jego
ruch. W takim przypadku możliwe jest dowolne początkowe położenie w układzie
współrzędnych (bądź na mapie), a wraz z ruchem robota kolejne obserwacje
doprowadzają do zawężenia przestrzeni możliwych rozwiązań, a po pewnym czasie
doprowadzają do podania jesnego, ostatecznego położenia. Takimi metodami są np.
różnego rodzaju filtry cząsteczkowe i inne metody probabilistyczne.


\subsubsection{Metody względne i bezwzględne}


\section{Tworzenie mapy}

\subsection{Rodzaje map}

\section{Generowanie trajektorii}

