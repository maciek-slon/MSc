% !TeX root = main.tex


\begin{savequote}[70mm]
,,''
\qauthor{}
\end{savequote}

\chapter{Nawigacja robotów mobilnych}
\label{chap:nawigacja}

Całe zadanie nawigacji robota mobilnego może zostać podzielone na kilka głównych
składników \cite[cz.~9]{szynkiewiczWR}:

\begin{itemize}
  \item precepcję,
  \item samolokalizację robota,
  \item wnioskowanie i/lub planowanie,
  \item tworzenie mapy otoczenia,
  \item generowanie trajektorii i omijanie przeszkód.
\end{itemize}

W zależności od potrzeb, nie wszystkie z nich muszą być uwzględnione w systemie
sterowania (np. roboty mogą poruszać się w terenie, którego mapa jest nieznana
i nie jest wymagane jej tworzenie, bądź też a priori zakłada się brak
przeszkód na trasie i nie uwzględnia problemu ich wykrywania). Niniejszy
rozdział opisuje ogólne metody realizacji poszczególnych elementów systemu.

\section{Percepcja}

Ta część systemu odpowiada za gromadzenie wiedzy o stanie zarówno samego robota,
jak i otaczającego go środowiska, w którym operuje. Dokonuje się tego przez
akwizycję danych z czujników a następnie wybieranie tych informacji, które mają
znaczenie w nawigacji.

\subsection{Klasyfikacja czujników}

Istnieje cały szereg różnych
czujników, które ogólnie można pogrupować ze względu na rodzaj mierzonych
wielkości (wyróżniamy proprioreceptory -- czujniki mierzące wartości parametrów
wewnętrznych robota,oraz eksteroreceptory -- mierzące wartości parametrów
otoczenia) oraz wpływ na otoczenie (tutaj mamy podział na czujniki pasywne i
aktywne)~\cite{siegwart}.


\subsubsection{Proprioreceptory i eksteroreceptory}

Proprioreceptory mierzą parametry określające w pewien sposób wewnętrzny stan
robota, takie jak prędkość silników czy napięcie akumulatorów. Wykorzystywane są
przy wyznaczaniu pozycji robota (czujniki odometryczne), sterowaniu (prędkość
obrotowa silników) czy diagnostyce jego podzespołów. Eksteroreceptory
pozyskują informację z otoczenia robota, takie jak natężenie
światła czy amplituda dźwięku. Można za ich pomocą mierzyć odległość,
orientację, lokalizować zewnętrzne obiekty aż wreszcie budować mapy.

Ważne jest to, że konkretne sensory są przypisywane do proprio- lub
eksteroreceptorów w zależności od ich zastosowania, a nie typu mierzonej
wartości (np. czujnik temperatury zamontowany na radiatorze silnika jest
czujnikiem wewnętrznym, a ten sam czujnik mierzący temperaturę otoczenia robota
jest już czujnikiem zewnętrznym).

\subsubsection{Pasywne i aktywne}

Czujniki pasywne rejestrują jeydnie energię przychodzącą z otoczenia i na
jej podstawie określają wielkość mierzonego parametru. Do takich czujników
należą termometry, mikrofony, kamery wizyjne, czy czujniki uderzeniowe.
Czujniki aktywne do wykonania pomiaru wymagają wyemitowania dodatkowej energii
do otoczenia, a następnie pomiar wykonywany jest na podstawie pomiaru ilości
energii, która powróciła do sensora. Zaletą czujników pasywnych jest brak
dodatkowych zakłóceń wprowadzanych do otoczenia, które mogą się pojawić
przy korzystaniu z czujników aktywnych (które mogą się pojawić przy korzystaniu
z wielu takich samych czujników na jednym bądź wielu robotach), problemem
przy czujnikach aktywnych jest także możliwość zafałszowania samej mierzonej
wartości przez emitowaną energię (np. powstawanie odblasków). Największą zaletą
czujników aktywnych jest możliwość pracy w trudniejszych warunkach (np. nie
jest wymagane oświetlenie zewnętrzne przy pomiarach odległości).

\subsection{Podstawowe klasy czujników}

Czujniki dostępne w robotyce można wreszcie podzielić na klasy ze względu
na metodę ich działania. poniżej przedstawione są główne klasy, w porządku
rosnącym pod względem skomplikowania konstrukcji, a malejącym ze względu na
dojrzałość konstrukcji~\cite{siegwart}.

\subsubsection{Czujniki dotykowe i zbliżeniowe}

Czujniki te mają za zadanie wykryć kontakt bądź niebezpiecznie małą odległość
od przedmiotów w otoczeniu robota, stosowane zarówno jako zderzaki do wykrywania
przeszkód jak i wyłączniki bezpieczeństwa (np. wykrywające nieprzewidziane
pojawienie się człowieka w zasięgu pracy robota). Mogą być zarówno w wersji
kontaktowej (przełączniki mechaniczne o różnej konstrukcji, od przycisków
do wąsów o pewnym stopniu podatności) jak i wykrywające przeszkody bez
kontaktu z nimi (np. bariery optyczne reagujące na przerwanie wiązki świetlnej
pomiędzy nadajnikiem i odbiornikiem, czy też indukcyjne czujniki zbliżeniowe
reagujące na pojawienie się w pobliżu metalowego przedmiotu).

\subsubsection{Czujniki odometryczne}

Czujniki odometryczne mierzą prędkość bądź położenie obrotowych elementów
robota (wału silnika, kół czy stawów). Odczyty z nich wykorzystywane są
później m.in. przy wyliczaniu pozycji robota. Do czujników tej klasy zalicza się
różnego typu enkodery (optyczne, magnetyczne itp.), resolwery, a także potencjometry.
Czujniki dotykowe i odometryczne są wykorzystywane jako podstawowe źródło
informacji w robotyce praktycznie w każdej kostrukcji, czy to robota mobilnego,
czy manipulatora.

\subsubsection{Czujniki orientacji}

Czujniki z tej grupy służą do mierzenia orientacji robotów (głównie mobilnych,
a przede wszystkim robotów latających) w pewnym globalnym układzie odniesienia.
Tutaj można wyróżnić trzy zasadnicze typy -- kompasy podające orientację
względem magnetycznego bieguna północnego Ziemi, żyroskopy służące do pomiaru
bądź utrzymywania położenia kątowego, oraz inklinometry służące do określania
kąta wychylenia od pionu wyznaczonego przez siłę grawitacji ziemskiej.

\subsubsection{Latarnie kierunkowe}

Do pomiaru położenia w globalnych układach współrzędnych służą czujniki
wykorzystujące triangulację. Zaliczamy do nich odbiorniki GPS, odbiorniki GSM
potrafiące lokalizować się na podstawie nadajników sieci komórkowych, a także
aktywne latarnie optyczne, radiowe bądź ultradźwiękowe.

\subsubsection{Aktywne czujniki odległości}

Do pomiaru odległości od przedmiotów otaczających robota, a nie będących
w bezpośrednim zasięgu czujników dotykowych, wykorzystuje się czujniki odległości
takie jak dalmierze i skanery laserowe, czujniki odbiciowe (zarówno na podczerwień,
jak i ultradźwiękowe), a także metody oparte o światło strukturalne (opisane szerzej
w poprzednim rozdziale).

\subsubsection{Czujniki prędkości i ruchu}

Do pomiaru prędkości robota względem zewnętrznego układu współrzędnych (bądź
względem innych obiektów w otoczeniu) wykorzystać można efekt Dopplera (zmiana
częstotliwości fali przy odbiciu od ruchomych obiektów, w szczgólności ruchomy
może być sam nadajnik, a mierzona zmiana częstotliwości fali odbitej od podłoża).
Najczęściej wykorzystywane są w robotach wodnych~\cite{doppler_underwater} i latających, gdzie nie
można zmierzyć prędkości w prosty sposób przy użyciu czujników odometrycznych
(nie jest znana prędkość powietrza/wody w otoczeniu robota)~\cite{whereami}.
Podejmowane były też testy z wykorzystaniem tego typu czujników przy lokalizacji
robotów mobilnych (rozwiązanie oparte o fale radiowe~\cite{doppler_mobilesensor}).
Czujnikami tego typu są sonary oraz radary dopplerowskie (ultradźwiękowe
oraz mikrofalowe).

\subsubsection{Czujniki wizyjne}

Najbardziej skomplikowane z punktu widzenia algorytmiki są czujniki wizyjne
(kamery oparte na sensorach CCD lub CMOS), działające w różnych pasmach
(zarówno w świetle widzialnym, jak i podczerwieni). Mogą być wykorzystywane
do biernego pomiaru odległości (np. stereowizja, bez wykorzystywania dodatkowych
emiterów, jak w przypadku światła strukturalnego), pomiarów prędkości,
a co za tym idzie położenia robota (algorytmy korzystające zarówno ze stereopary
kamer~\cite{vodom_stereo} jak i kamery pojedynczej~\cite{vodom_mono}) czy budowania
map (zamierających punkty charakterystyczne otoczenia~\cite{vslam}). Oprócz
czynności czysto nawigacyjnych sensory tego typu wykorzystywane są przy całej
gamie innych czynności, jak wykrywanie i śledzenie obiektów w obszarze pracy robota,
rozpoznawanie twarzy w aplikacjach wymagających bliskej współpracy z operatorem
czy wreszcie całościowej analizy i rozumienia sceny.

\subsubsection{Czujniki innych wielkości fizycznych i chemicznych}

Czujniki wielkości fizycznych i chemicznych trudno dopasować do przedstawionego
wyżej schematu (uporządkowania rosnącego ze względu na stopień skomplikowania).
W tej kategorii mieszczą się zarówno proste sensory takich wartości jak
napięcie (i innych, które można łatwo przekształcić na napięcie, takich jak
wiele wielkości elektrycznych czy temperatura), przez bardziej skomplikowane
czujniki chemiczne mierzące stężenia różnych substacji w otoczeniu (zarówno
wodzie jak i powietrzu), aż po skomplikowane sensory mierzące zapachy
w specjalistycznych aplikacjach robotycznych.

\section{Samolokalizacja}

Pod pojęciem samolokalizacji robota rozumie się określenie bieżącej pozycji
robota w pewnym, ustalonym układzie odniesienia. Do obliczeń wykorzystywane mogą
być dane uzyskiwane z czujników robota,

\subsection{Klasyfikacja metod lokalizacji}

Metody lokalizacji robotów mobilnych można podzielić biorąc pod uwagę różne
kryteria. Kilka możliwych podziałów przedstawione jest w dalszej części
rozdziału, wraz z przykładami.

\subsubsection{Metody lokalne i globalne}

W zależności od lokalizacji układu odniesienia metody lokalizacji możemy
podzielić na lokalne i globalne. W metodach lokalnych układ odniesienia, w
którym następuje lokalizacja jest związany z punktem, z którego robot rozpoczyna
pracę (a więc początkowe położenie robota jest zerowe bądź wskazane przez
operatora), tak więc rozpoczynając pracę z faktycznie różnych punktów i
wykonując tę samą trajektorię (zakładając, że w ogóle może ją w każdym
przypadku wykonać), mimo takiej samej pozycji wyliczonej przez układ
lokalizacji robot będzie się znajdował w różnych miejscach.

Metody globalne operują z kolei w układach odniesienia związanych z elementami,
które nie są bezpośrednio zależne od robota. Mogą to być układy związane z danym
pomieszczeniem (z ustalonym odgórnie punktem zerowym) bądź takie jak układ
współrzędnych geograficznych, jak również oparte o mapę i lokalizujące robota
względem umieszczonych na niej obiektów. W takich metodach kluczowe jest
określenie pozycji początkowej, które nie zawsze jest łatwe do wykonania.
Korzystając z odbiorników GPS łatwo zlokalizować robota w przestrzeni otwartej,
trudniej jest to wykonać w pomieszczeniach. Rozwiązują to metody czynne, których
opis znajduje się w dalszej części rozdziału.

\subsubsection{Środowisko statyczne i  dynamiczne}

Poprzez określenie środowiska statycznego rozumie się takie otoczenie, w którym
ruch innych obiektów oprócz robota nie występuje bądź jest niewykrywalny przez
jego czujniki. Podczas pracy w środowisku dynamicznym (a więc takim, gdzie
pomimo braku ruchu robota czujniki rejestrują ruch przeszkód) algorytmy
lokalizacji muszą uwzględniać ten fakt i stosować różne rodzaje filtrowania
danych tak, aby ruch przeszkód wokół robota nie powodował zmian jego wyliczonego
położenia. Metodami odpornymi na zmiany w otoczeniu robota są wszystkie te,
które wykorzystują informacje o przesunięciu odczytywane z czujników
przesunięcia (np. odometryczne bądź inercyjne), wrażliwe mogą być z kolei metody
wykorzystujące odczyty laserowe.

\subsubsection{Metody bierne i czynne}

Przy biernych metodach lokalizacji do określenia pozycji robota jego ruch
(a dokładniej ruch sensorów) nie jest konieczny. Tak działa chociażby GPS, oraz
wszystkie metody wykorzystujące zewnętrzne nadajniki (np. latarnie radiowe). W
metodach czynnych do dokładnego określenia pozycji robota wymagany jest jego
ruch. W takim przypadku możliwe jest dowolne początkowe położenie w układzie
współrzędnych (bądź na mapie), a wraz z ruchem robota kolejne obserwacje
doprowadzają do zawężenia przestrzeni możliwych rozwiązań, a po pewnym czasie
doprowadzają do podania jesnego, ostatecznego położenia. Takimi metodami są np.
różnego rodzaju filtry cząsteczkowe i inne metody probabilistyczne.

\section{Mapy otoczenia}

W przypadku, kiedy konieczna jest znajomość otoczenia robota wymagany jest pewien
sposób jej reprezentacji, taki, który będzie można stosunkowo łatwo analizować
i podejmować na jego podstawie decyzje (np. o aktualnej pozycji). Wyróżnić
można dwa podstawowe zagadnienia w tym zakresie -- sam sposób reprezentacji,
oraz metody tworzenia map.

\subsection{Rodzaje map}

Problem odwzorowania otoczenia robota na postać użyteczną w jego sterowaniu
związany z ogólnymi założeniami przyjętymi co do dokładności lokalizacji robota
na owej mapie oraz precyzji osiągania celów. Im większa wymagana precyzja
tym dokładniejsze muszą być wykorzystywane mapy. Kolejnym czynnikiem wpływającym
na wybór typu mapy są czujniki dostępne na robocie -- bezcelowe jest wykorzystywanie
map przechowujących charakterystyczne punkty otoczenia uzyskane np. w algorytmie
vSLAM na robocie, który nie jest wyposażony w kamery. Ostatnim czynnikiem
jest możliwy do poświęcenia czas procesora wymagany do obsługi samych map
i lokalizacji robota.

\subsubsection{Mapy metryczne}

Mapy metryczne mają ustalony układ współrzędnych, w którym są tworzone, odpowiadają
klasycznym mapom spotykanym w życiu codziennym (np. plany miast, mapy turystyczne).
Mapy metryczne w postaci ciągłej (a więc dokładnie opisujące rozmiar i położenie
przeszkód w otoczeniu) mogą być tworzone jedynie w przypadku niewielkiej ilości
i złożoności opisywanych obiektów. Praktycznie stosowanie takich map jest niewygodne,
aby zmniejszyć ilość wymaganych danych stosowane są różne metody ich przybliżania.
Jedną z metod jest zastępowanie obiektów na mapie ich uproszczonymi odpowiednikami
(np. podstawowymi fogurami geometrycznymi i ich złożeniami czy opisywanie ich za pomocą
krzywych parametrycznych, np. NURBS). W dalszym etapie można dokonać podziału mapy
na segmenty odpowiadające obszarom wolnym (opisane przy pomocy wielokątów wypukłych,
np. trapezów i trójkątów) wraz z załączoną informacją o ich sąsiedztwie.

Inną metodą jest dyskretyzacja całej przestrzeni mapy. Można to wykonać przy pomocy
równomiernej siatki (np. kwadratów), otrzymując mapę złożoną z jednakowych komórek,
z których każda może przechowywać informację o jej zajętości (bądź binarnie wolna/zajęta,
bądź z pewnymi dodatkowymi wartościami mogącymi reprezentować np. prawdopodobieństwo
obecności przeszkody w danym miejscu). Zaletą takiej mapy jej prostota implementacji
(podstawowa struktura danych oparta o tablicę wielowymiarową), ogólność (można opisywać
przestrzeń bez żadnych dodatkowych założeń co do przeszkód) oraz możliwość łatwej modyfikacji
w trakcie działania (dodanie czy usunięcie przeszkody na mapie nie wymaga przebudowy
całej struktury, a jedynie zmiany wartości w komórce). Wielkość komórek decyduje wprost
o wymaganiach pamięciowych do przechowywania mapy, wpływa też wprost na możliwość
wykrywania wąskich przejść pomiędzy przeszkodami -- jeśli komórki będą zbyt duże, to
drobne przeszkody mogą się ze sobą zlewać. Tego rodzaju mapy są często wykorzystywane
przy przechowywaniu map zajętości tworzonych dynamicznie w trakcie ruchu robota.
Inną metodą dyskretyzacji jest wykorzystanie metod adaptacyjnych, np. drzew czwórkowych
w przypadku map dwuwymiarowych. Podział komórki następuje wtedy, gdy nie jest ona
ani całkowicie wolna, ani całkowicie zajęta przez przeszkodę. W ten sposób, ustalając
dodatkowo pewną maksymalną głębokość podziału, otrzymujemy mapę dobrze dopasowaną
zarówno do dużych jak i drobnych przeszkód, dodatkowo zachowującą wszystkie wąskie
przejścia,, a także bardzo efektywną z punktu widzenia wymagań pamieciowych (sąsiadujące
komórki jednakowego typu są przechowywane jako pojedynczy węzeł). Problemem jest
nieco bardziej skomplikowana struktura danych (najczęściej drzewiasta), w której
operacje dodawania i usuwania przeszkód wymagają przeprowadzenia procesu podziału
od początku, jeśli nie w calej to często w dużej części mapy. Trudniejsza jest też
reprezentacja prawdopodobieństwa pojawienia się przeszkód na mapie -- z założenia kryterium
podziału jest binarne, tak samo jak przechowywane w wezłach wartości.

\subsubsection{Mapy topologiczne}

Mapy topologiczne zamiast informacji o całym otoczeniu zawierają jedynie wiedzę
o charakterystycznych obiektach, które mogą być reprezentatywne dla danego podobszaru.
Reprezentowane są najczęściej w postaci grafu, gdzie węzły reprezentują podobszary
(przechowując też informację o ich cechach charakterystycznych), natomiast łuki
pomiędzy węzłami określają możliwe przejścia pomiędzy nimi. Ważne jest to, że nie
ma nigdzie zapisanej wprost informacji o wolnej bądź zajętej przestrzeni na mapie,
a także o fizycznych właściwościach (np. wielkości pokoju reprezentowanego przez węzeł,
chyba, że owa wielkość będzie cechą rozpoznawczą podobszaru). Przykładem takiej mapy
w świecie rzeczywistym jest np. schemat połączeń komunikacyjnych w mieście -- węzły (przystanki)
określone są nazwami, łuki określają możliwe do przebycia komunikacją odcinki pomiędzy
nimi.

Przy używaniu map togologicznych kluczowe jest to, aby robot był w stanie zlokalizować
się jednoznacznie w danym węźle i wykryć charakterystyczne elementy w otoczeniu.
Największą ich zaletą jest duża efektywność obliczeniowa i skalowalność, gdyż rozmiar
reprezentacji nie zależy od rozmiaru obszaru do opisania, a jedynie od ilości wymaganych,
rozróżnialnych obszarów (a nawet przy dużej ich liczbie każdy z nich opisany jest
stosunkowo niewielką liczbą danych). Wiąże się z tym też wada takich map -- lokalizacja
przy ich użyciu jest jedynie zgrubna, dokładna pozycja w danym podobszarze jest nieznana.

\subsubsection{Mapy hybrydowe}

Mapy topologiczne łączą cechy map metrycznych z topologicznymi, czerpiąc zalety każdej z nich.
Duże obszary reprezentowane są przy pomocy grafów (np. cały teren fabryki), a mniejsze
podobszary (np. konkretne budynki i pomieszczenia w nich) opisywane są przy pomocy
map geometrycznych. Dzięki temu nie ma konieczności przechowywania jednej, dużej
mapy metrycznej, a każda z mniejszych map może być dokładniejsza, umożliwia to też
sprawniejszą nawigację w trudnych obszarach o skomplikowanej strukturze (labirynt
łatwiej przedstawić na mapie geometrycznej niż opisywać go grafem tak, aby robot potrafił
wyszukać w nim punkty charakterystyczne).

\subsection{Tworzenie map}

\subsubsection{Wykorzystanie gotowych map}

\subsubsection{Ręczne tworzenie prostych map otoczenia}

\subsubsection{Automatyczne tworzenie map przez robota}

\section{Generowanie trajektorii}

