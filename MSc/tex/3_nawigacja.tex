% !TeX root = main.tex


\begin{savequote}[70mm]
,,''
\qauthor{}
\end{savequote}

\chapter{Nawigacja robotów mobilnych}
\label{chap:nawigacja}

Całe zadanie nawigacji robota mobilnego może zostać podzielone na kilka głównych
składników \cite[cz.~9]{szynkiewiczWR}:

\begin{itemize}
  \item precepcję,
  \item samolokalizację robota,
  \item wnioskowanie i/lub planowanie,
  \item tworzenie mapy otoczenia,
  \item generowanie trajektorii i omijanie przeszkód.
\end{itemize}

W zależności od potrzeb, nie wszystkie z nich muszą być uwzględnione w systemie
sterowania (np. roboty mogą poruszać się w terenie, którego mapa jest nieznana
i nie jest wymagane jej tworzenie, bądź też a priori zakłada się brak
przeszkód na trasie i nie uwzględnia problemu ich wykrywania). Niniejszy
rozdział opisuje ogólne metody realizacji poszczególnych elementów systemu.

\section{Percepcja}

Ta część systemu odpowiada za gromadzenie wiedzy o stanie zarówno samego robota,
jak i otaczającego go środowiska, w którym operuje. Dokonuje się tego przez
akwizycję danych z czujników a następnie wybieranie tych informacji, które mają
znaczenie w nawigacji.

\subsection{Klasyfikacja czujników}

Istnieje cały szereg różnych
czujników, które ogólnie można pogrupować ze względu na rodzaj mierzonych
wielkości (wyróżniamy proprioreceptory -- czujniki mierzące wartości parametrów
wewnętrznych robota,oraz eksteroreceptory -- mierzące wartości parametrów
otoczenia) oraz wpływ na otoczenie (tutaj mamy podział na czujniki pasywne i
aktywne)~\cite{siegwart}.


\subsubsection{Proprioreceptory i eksteroreceptory}

Proprioreceptory mierzą parametry określające w pewien sposób wewnętrzny stan
robota, takie jak prędkość silników czy napięcie akumulatorów. Wykorzystywane są
przy wyznaczaniu pozycji robota (czujniki odometryczne), sterowaniu (prędkość
obrotowa silników) czy diagnostyce jego podzespołów. Eksteroreceptory
pozyskują informację z otoczenia robota, takie jak natężenie
światła czy amplituda dźwięku. Można za ich pomocą mierzyć odległość,
orientację, lokalizować zewnętrzne obiekty aż wreszcie budować mapy.

Ważne jest to, że konkretne sensory są przypisywane do proprio- lub
eksteroreceptorów w zależności od ich zastosowania, a nie typu mierzonej
wartości (np. czujnik temperatury zamontowany na radiatorze silnika jest
czujnikiem wewnętrznym, a ten sam czujnik mierzący temperaturę otoczenia robota
jest już czujnikiem zewnętrznym).

\subsubsection{Pasywne i aktywne}

Czujniki pasywne rejestrują jeydnie energię przychodzącą z otoczenia i na
jej podstawie określają wielkość mierzonego parametru. Do takich czujników
należą termometry, mikrofony, kamery wizyjne, czy czujniki uderzeniowe.
Czujniki aktywne do wykonania pomiaru wymagają wyemitowania dodatkowej energii
do otoczenia, a następnie pomiar wykonywany jest na podstawie pomiaru ilości
energii, która powróciła do sensora. Zaletą czujników pasywnych jest brak
dodatkowych zakłóceń wprowadzanych do otoczenia, które mogą się pojawić
przy korzystaniu z czujników aktywnych (które mogą się pojawić przy korzystaniu
z wielu takich samych czujników na jednym bądź wielu robotach), problemem
przy czujnikach aktywnych jest także możliwość zafałszowania samej mierzonej
wartości przez emitowaną energię (np. powstawanie odblasków). Największą zaletą
czujników aktywnych jest możliwość pracy w trudniejszych warunkach (np. nie
jest wymagane oświetlenie zewnętrzne przy pomiarach odległości).

\subsection{Podstawowe klasy czujników}

Czujniki dostępne w robotyce można wreszcie podzielić na klasy ze względu
na metodę ich działania. poniżej przedstawione są główne klasy, w porządku
rosnącym pod względem skomplikowania konstrukcji, a malejącym ze względu na
dojrzałość konstrukcji~\cite{siegwart}.

\subsubsection{Czujniki dotykowe i zbliżeniowe}

Czujniki te mają za zadanie wykryć kontakt bądź niebezpiecznie małą odległość
od przedmiotów w otoczeniu robota, stosowane zarówno jako zderzaki do wykrywania
przeszkód jak i wyłączniki bezpieczeństwa (np. wykrywające nieprzewidziane
pojawienie się człowieka w zasięgu pracy robota). Mogą być zarówno w wersji
kontaktowej (przełączniki mechaniczne o różnej konstrukcji, od przycisków
do wąsów o pewnym stopniu podatności) jak i wykrywające przeszkody bez
kontaktu z nimi (np. bariery optyczne reagujące na przerwanie wiązki świetlnej
pomiędzy nadajnikiem i odbiornikiem, czy też indukcyjne czujniki zbliżeniowe
reagujące na pojawienie się w pobliżu metalowego przedmiotu).

\subsubsection{Czujniki odometryczne}

Czujniki odometryczne mierzą prędkość bądź położenie obrotowych elementów
robota (wału silnika, kół czy stawów). Odczyty z nich wykorzystywane są
później m.in. przy wyliczaniu pozycji robota. Do czujników tej klasy zalicza się
różnego typu enkodery (optyczne, magnetyczne itp.), resolwery, a także potencjometry.
Czujniki dotykowe i odometryczne są wykorzystywane jako podstawowe źródło
informacji w robotyce praktycznie w każdej kostrukcji, czy to robota mobilnego,
czy manipulatora.

\subsubsection{Czujniki orientacji}

Czujniki z tej grupy służą do mierzenia orientacji robotów (głównie mobilnych,
a przede wszystkim robotów latających) w pewnym globalnym układzie odniesienia.
Tutaj można wyróżnić trzy zasadnicze typy -- kompasy podające orientację
względem magnetycznego bieguna północnego Ziemi, żyroskopy służące do pomiaru
bądź utrzymywania położenia kątowego, oraz inklinometry służące do określania
kąta wychylenia od pionu wyznaczonego przez siłę grawitacji ziemskiej.

\subsubsection{Latarnie kierunkowe}

Do pomiaru położenia w globalnych układach współrzędnych służą czujniki
wykorzystujące triangulację. Zaliczamy do nich odbiorniki GPS, odbiorniki GSM
potrafiące lokalizować się na podstawie nadajników sieci komórkowych, a także
aktywne latarnie optyczne, radiowe bądź ultradźwiękowe.

\subsubsection{Aktywne czujniki odległości}

Do pomiaru odległości od przedmiotów otaczających robota, a nie będących
w bezpośrednim zasięgu czujników dotykowych, wykorzystuje się czujniki odległości
takie jak dalmierze i skanery laserowe, czujniki odbiciowe (zarówno na podczerwień,
jak i ultradźwiękowe), a także metody oparte o światło strukturalne (opisane szerzej
w poprzednim rozdziale).

\subsubsection{Czujniki prędkości i ruchu}

Do pomiaru prędkości robota względem zewnętrznego układu współrzędnych (bądź
względem innych obiektów w otoczeniu) wykorzystać można efekt Dopplera (zmiana
częstotliwości fali przy odbiciu od ruchomych obiektów, w szczgólności ruchomy
może być sam nadajnik, a mierzona zmiana częstotliwości fali odbitej od podłoża).
Najczęściej wykorzystywane są w robotach wodnych~\cite{doppler_underwater} i latających, gdzie nie
można zmierzyć prędkości w prosty sposób przy użyciu czujników odometrycznych
(nie jest znana prędkość powietrza/wody w otoczeniu robota)~\cite{whereami}.
Podejmowane były też testy z wykorzystaniem tego typu czujników przy lokalizacji
robotów mobilnych (rozwiązanie oparte o fale radiowe~\cite{doppler_mobilesensor}).
Czujnikami tego typu są sonary oraz radary dopplerowskie (ultradźwiękowe
oraz mikrofalowe).

\subsubsection{Czujniki wizyjne}

Najbardziej skomplikowane z punktu widzenia algorytmiki są czujniki wizyjne
(kamery oparte na sensorach CCD lub CMOS), działające w różnych pasmach
(zarówno w świetle widzialnym, jak i podczerwieni). Mogą być wykorzystywane
do biernego pomiaru odległości (np. stereowizja, bez wykorzystywania dodatkowych
emiterów, jak w przypadku światła strukturalnego), pomiarów prędkości,
a co za tym idzie położenia robota (algorytmy korzystające zarówno ze stereopary
kamer~\cite{vodom_stereo} jak i kamery pojedynczej~\cite{vodom_mono}) czy budowania
map (zamierających punkty charakterystyczne otoczenia~\cite{vslam}). Oprócz
czynności czysto nawigacyjnych sensory tego typu wykorzystywane są przy całej
gamie innych czynności, jak wykrywanie i śledzenie obiektów w obszarze pracy robota,
rozpoznawanie twarzy w aplikacjach wymagających bliskej współpracy z operatorem
czy wreszcie całościowej analizy i rozumienia sceny.

\subsubsection{Czujniki innych wielkości fizycznych i chemicznych}

Czujniki wielkości fizycznych i chemicznych trudno dopasować do przedstawionego
wyżej schematu (uporządkowania rosnącego ze względu na stopień skomplikowania).
W tej kategorii mieszczą się zarówno proste sensory takich wartości jak
napięcie (i innych, które można łatwo przekształcić na napięcie, takich jak
wiele wielkości elektrycznych czy temperatura), przez bardziej skomplikowane
czujniki chemiczne mierzące stężenia różnych substacji w otoczeniu (zarówno
wodzie jak i powietrzu), aż po skomplikowane sensory mierzące zapachy
w specjalistycznych aplikacjach robotycznych.

\section{Samolokalizacja}

Pod pojęciem samolokalizacji robota rozumie się określenie bieżącej pozycji
robota w pewnym, ustalonym układzie odniesienia. Do obliczeń wykorzystywane mogą
być dane uzyskiwane z czujników robota,

\subsection{Klasyfikacja metod lokalizacji}

Metody lokalizacji robotów mobilnych można podzielić biorąc pod uwagę różne
kryteria. Kilka możliwych podziałów przedstawione jest w dalszej części
rozdziału, wraz z przykładami.

\subsubsection{Metody lokalne i globalne}

W zależności od lokalizacji układu odniesienia metody lokalizacji możemy
podzielić na lokalne i globalne. W metodach lokalnych układ odniesienia, w
którym następuje lokalizacja jest związany z punktem, z którego robot rozpoczyna
pracę (a więc początkowe położenie robota jest zerowe bądź wskazane przez
operatora), tak więc rozpoczynając pracę z faktycznie różnych punktów i
wykonując tę samą trajektorię (zakładając, że w ogóle może ją w każdym
przypadku wykonać), mimo takiej samej pozycji wyliczonej przez układ
lokalizacji robot będzie się znajdował w różnych miejscach.

Metody globalne operują z kolei w układach odniesienia związanych z elementami,
które nie są bezpośrednio zależne od robota. Mogą to być układy związane z danym
pomieszczeniem (z ustalonym odgórnie punktem zerowym) bądź takie jak układ
współrzędnych geograficznych, jak również oparte o mapę i lokalizujące robota
względem umieszczonych na niej obiektów. W takich metodach kluczowe jest
określenie pozycji początkowej, które nie zawsze jest łatwe do wykonania.
Korzystając z odbiorników GPS łatwo zlokalizować robota w przestrzeni otwartej,
trudniej jest to wykonać w pomieszczeniach. Rozwiązują to metody czynne, których
opis znajduje się w dalszej części rozdziału.

\subsubsection{Środowisko statyczne i  dynamiczne}

Poprzez określenie środowiska statycznego rozumie się takie otoczenie, w którym
ruch innych obiektów oprócz robota nie występuje bądź jest niewykrywalny przez
jego czujniki. Podczas pracy w środowisku dynamicznym (a więc takim, gdzie
pomimo braku ruchu robota czujniki rejestrują ruch przeszkód) algorytmy
lokalizacji muszą uwzględniać ten fakt i stosować różne rodzaje filtrowania
danych tak, aby ruch przeszkód wokół robota nie powodował zmian jego wyliczonego
położenia. Metodami odpornymi na zmiany w otoczeniu robota są wszystkie te,
które wykorzystują informacje o przesunięciu odczytywane z czujników
przesunięcia (np. odometryczne bądź inercyjne), wrażliwe mogą być z kolei metody
wykorzystujące odczyty laserowe.

\subsubsection{Metody bierne i czynne}

Przy biernych metodach lokalizacji do określenia pozycji robota jego ruch
(a dokładniej ruch sensorów) nie jest konieczny. Tak działa chociażby GPS, oraz
wszystkie metody wykorzystujące zewnętrzne nadajniki (np. latarnie radiowe). W
metodach czynnych do dokładnego określenia pozycji robota wymagany jest jego
ruch. W takim przypadku możliwe jest dowolne początkowe położenie w układzie
współrzędnych (bądź na mapie), a wraz z ruchem robota kolejne obserwacje
doprowadzają do zawężenia przestrzeni możliwych rozwiązań, a po pewnym czasie
doprowadzają do podania jesnego, ostatecznego położenia. Takimi metodami są np.
różnego rodzaju filtry cząsteczkowe i inne metody probabilistyczne.

\section{Mapy otoczenia}

W przypadku, kiedy konieczna jest znajomość otoczenia robota wymagany jest pewien
sposób jej reprezentacji, taki, który będzie można stosunkowo łatwo analizować
i podejmować na jego podstawie decyzje (np. o aktualnej pozycji). Wyróżnić
można dwa podstawowe zagadnienia w tym zakresie -- sam sposób reprezentacji,
oraz metody tworzenia map.

\subsection{Rodzaje map}

Problem odwzorowania otoczenia robota na postać użyteczną w jego sterowaniu
związany z ogólnymi założeniami przyjętymi co do dokładności lokalizacji robota
na owej mapie oraz precyzji osiągania celów. Im większa wymagana precyzja
tym dokładniejsze muszą być wykorzystywane mapy. Kolejnym czynnikiem wpływającym
na wybór typu mapy są czujniki dostępne na robocie -- bezcelowe jest wykorzystywanie
map przechowujących charakterystyczne punkty otoczenia uzyskane np. w algorytmie
vSLAM na robocie, który nie jest wyposażony w kamery. Ostatnim czynnikiem
jest możliwy do poświęcenia czas procesora wymagany do obsługi samych map
i lokalizacji robota.

\subsubsection{Mapy metryczne}

Mapy metryczne mają ustalony układ współrzędnych, w którym są tworzone, odpowiadają
klasycznym mapom spotykanym w życiu codziennym (np. plany miast, mapy turystyczne).
Mapy metryczne w postaci ciągłej (a więc dokładnie opisujące rozmiar i położenie
przeszkód w otoczeniu) mogą być tworzone jedynie w przypadku niewielkiej ilości
i złożoności opisywanych obiektów. Praktycznie stosowanie takich map jest niewygodne,
aby zmniejszyć ilość wymaganych danych stosowane są różne metody ich przybliżania.
Jedną z metod jest zastępowanie obiektów na mapie ich uproszczonymi odpowiednikami
(np. podstawowymi fogurami geometrycznymi i ich złożeniami czy opisywanie ich za pomocą
krzywych parametrycznych, np. NURBS). W dalszym etapie można dokonać podziału mapy
na segmenty odpowiadające obszarom wolnym (opisane przy pomocy wielokątów wypukłych,
np. trapezów i trójkątów) wraz z załączoną informacją o ich sąsiedztwie.

Inną metodą jest dyskretyzacja całej przestrzeni mapy. Można to wykonać przy pomocy
równomiernej siatki (np. kwadratów), otrzymując mapę złożoną z jednakowych komórek,
z których każda może przechowywać informację o jej zajętości (bądź binarnie wolna/zajęta,
bądź z pewnymi dodatkowymi wartościami mogącymi reprezentować np. prawdopodobieństwo
obecności przeszkody w danym miejscu). Zaletą takiej mapy jej prostota implementacji
(podstawowa struktura danych oparta o tablicę wielowymiarową), ogólność (można opisywać
przestrzeń bez żadnych dodatkowych założeń co do przeszkód) oraz możliwość łatwej modyfikacji
w trakcie działania (dodanie czy usunięcie przeszkody na mapie nie wymaga przebudowy
całej struktury, a jedynie zmiany wartości w komórce). Wielkość komórek decyduje wprost
o wymaganiach pamięciowych do przechowywania mapy, wpływa też wprost na możliwość
wykrywania wąskich przejść pomiędzy przeszkodami -- jeśli komórki będą zbyt duże, to
drobne przeszkody mogą się ze sobą zlewać. Tego rodzaju mapy są często wykorzystywane
przy przechowywaniu map zajętości tworzonych dynamicznie w trakcie ruchu robota.
Inną metodą dyskretyzacji jest wykorzystanie metod adaptacyjnych, np. drzew czwórkowych
w przypadku map dwuwymiarowych. Podział komórki następuje wtedy, gdy nie jest ona
ani całkowicie wolna, ani całkowicie zajęta przez przeszkodę. W ten sposób, ustalając
dodatkowo pewną maksymalną głębokość podziału, otrzymujemy mapę dobrze dopasowaną
zarówno do dużych jak i drobnych przeszkód, dodatkowo zachowującą wszystkie wąskie
przejścia,, a także bardzo efektywną z punktu widzenia wymagań pamieciowych (sąsiadujące
komórki jednakowego typu są przechowywane jako pojedynczy węzeł). Problemem jest
nieco bardziej skomplikowana struktura danych (najczęściej drzewiasta), w której
operacje dodawania i usuwania przeszkód wymagają przeprowadzenia procesu podziału
od początku, jeśli nie w calej to często w dużej części mapy. Trudniejsza jest też
reprezentacja prawdopodobieństwa pojawienia się przeszkód na mapie -- z założenia kryterium
podziału jest binarne, tak samo jak przechowywane w wezłach wartości.

\subsubsection{Mapy topologiczne}

Mapy topologiczne zamiast informacji o całym otoczeniu zawierają jedynie wiedzę
o charakterystycznych obiektach, które mogą być reprezentatywne dla danego podobszaru.
Reprezentowane są najczęściej w postaci grafu, gdzie węzły reprezentują podobszary
(przechowując też informację o ich cechach charakterystycznych), natomiast łuki
pomiędzy węzłami określają możliwe przejścia pomiędzy nimi. Ważne jest to, że nie
ma nigdzie zapisanej wprost informacji o wolnej bądź zajętej przestrzeni na mapie,
a także o fizycznych właściwościach (np. wielkości pokoju reprezentowanego przez węzeł,
chyba, że owa wielkość będzie cechą rozpoznawczą podobszaru). Przykładem takiej mapy
w świecie rzeczywistym jest np. schemat połączeń komunikacyjnych w mieście -- węzły (przystanki)
określone są nazwami, łuki określają możliwe do przebycia komunikacją odcinki pomiędzy
nimi.

Przy używaniu map togologicznych kluczowe jest to, aby robot był w stanie zlokalizować
się jednoznacznie w danym węźle i wykryć charakterystyczne elementy w otoczeniu.
Największą ich zaletą jest duża efektywność obliczeniowa i skalowalność, gdyż rozmiar
reprezentacji nie zależy od rozmiaru obszaru do opisania, a jedynie od ilości wymaganych,
rozróżnialnych obszarów (a nawet przy dużej ich liczbie każdy z nich opisany jest
stosunkowo niewielką liczbą danych). Wiąże się z tym też wada takich map -- lokalizacja
przy ich użyciu jest jedynie zgrubna, dokładna pozycja w danym podobszarze jest nieznana.

\subsubsection{Mapy hybrydowe}

Mapy topologiczne łączą cechy map metrycznych z topologicznymi, czerpiąc zalety każdej z nich.
Duże obszary reprezentowane są przy pomocy grafów (np. cały teren fabryki), a mniejsze
podobszary (np. konkretne budynki i pomieszczenia w nich) opisywane są przy pomocy
map geometrycznych. Dzięki temu nie ma konieczności przechowywania jednej, dużej
mapy metrycznej, a każda z mniejszych map może być dokładniejsza, umożliwia to też
sprawniejszą nawigację w trudnych obszarach o skomplikowanej strukturze (labirynt
łatwiej przedstawić na mapie geometrycznej niż opisywać go grafem tak, aby robot potrafił
wyszukać w nim punkty charakterystyczne).

\subsection{Tworzenie map}

Żeby w ogóle móc wykorzystać mapę przy nawigacji robota trzeba ją posiadać.
Istnieje kilka dróg, które pozwalają na uzyskanie mapy interesującego obszaru,
od wykorzystania gotowych map, przez ręczne ich tworzenie na podstawie pewnych
pomiarów, aż do automatycznego ich generowania przy pomocy specjalizowanych
aplikacji uruchomionych na robocie.

\subsubsection{Wykorzystanie gotowych map}

Jeśli środowisko, w którym będzie poruszał się robot jest w pewien sposób
ustrukturyzowane oraz opisane, można wykorzystać istniejące już mapy w celu
nawigacji i lokalizacji. Mogą one wymagać mniejszych bądź większych ingerencji
w celu dopasowania sposobu reprezentacji czy zmniejszenia szczegółowości,
jest to jednak metoda wymagająca stosunkowo mało nakładów pracy przy tworzeniu.
Przykładem gotowej mapy możliwej do wykorzystania może być plan budynku,
z zaznacoznymi ścianami i drzwiami. Konieczne może okazać się wyczyszczenie
z niej symboli samych drzwi, gdyż mogą być interpretowane przez algorytm
jako przeszkody, jednak są dość dokładne i łatwe do zdobycia.

\subsubsection{Ręczne tworzenie prostych map otoczenia}

Sposób opisany w poprzednim punkcie pozwala na pozyskanie map w postaci metrycznej,
dopasowanie istniejących map topologicznych wymaga większych nakładów pracy
i nie wystarcza zwykły skan. W tym wypadku cała mapa praktycznie musi być
stworzona od nowa, bazując jedynie na informacjach z mapy oryginalnej, wymagana
jest całkowita zmiana sposobu reprezentacji. Mapy metryczne również mogą być
tworzone ręcznie, zarówno w programach do grafiki rastrowej (modelowanie map
w postaci dyskretnej siatki) jak i wektorowej (mapy reprezentowane przy pomocy
różnych form geometrycznych). Przy ich tworzeniu można korzystać z map istniejących
jako punktu odniesienia, a uzyskane wyniki są z reguły w większym stopniu
dopasowane do wymagań algorytmów nawigacji.

\subsubsection{Automatyczne tworzenie map przez robota}

Ostatnią metodą jest wreszcie automatyczne tworzenie map otoczenia. Ogólnie
algorytmy tego typu składają się z fazy akwizycji danych oraz następującej
po niej fazy tworzenia mapy (dopasowania kolejnych pomiarów, oceny jakości
dopasowania itp). Rozwiązanie takie pozwalają na zastosowanie bardziej skomplikowanych
algorytmów dokładniej odwzorowujących otoczenie robota. Czasami jednak konieczne
jest tworzenie mapy on-line, inkrementalnie w trakcie działania robota,
wraz z jednoczesną lokalizacją. Cała grupa takich algorytmów zwana jest SLAM
(skrót od angielskiego Simultaneous Localization and Mapping, jednoczesna
lokalizacja i tworzenie mapy). Początkowo robot startuje z pozycji zerowej,
nie posiadając żadnej informacji o otoczeniu. Następnie porusza się (kierowany przez
operatora bądź korzystając ze swoich czujników w celu omijania przeszkód) zbierając
kolejne odczyty z sensora odpowiedzialnego za budowanie mapy (może być to skaner laserowy~\cite{laser_slam},
różnego rodzaju inne czujniki odległości czy kamera bądź ich zestaw~\cite{vslam}). Dodatkową
informacją, którą można podać do algorytmu jest przybliżone przemieszczenie robota
pomiędzy pomiarami (uzyskiwane np. z odometrii), jego podanie nie jest jednak w ogólności
konieczne, parametr ten może zostać wyliczony przez porównywanie kolejnych odczytów
(wtedy jednak algorytm jest dużo bardziej obciążający dla systemu i trudniejszy do zrealizowania
w czasie rzeczywistym). Agregacja kolejnych odczytów wraz z przybliżoną estymacją
pozycji robota pozwalają na inkrementalne tworzenie mapy otoczenia oraz wprowadzanie
poprawek na pozycję robota w układzie współrzędnych z nią związanym \cite{Dissanayake01asolution}.
Dodatkowo do poprawy jakości tworzonej mapy (np. niwelację kumulujących się błędów orientacji
wynikających z optymalizacyjnego charakteru algorytmów) wykorzystuje się wykrywanie pętli
w ścieżce robota~\cite{slam-loop}. Jeśli podczas tworzenia mapy zostanie wykryte
miejsce, które wczesniej zostało odwiedzone (na podstawie różnych kryteriów, np. charakterystycznych
elementów w obrazie), to wprowadzana jest taka poprawka, aby obecna pozycja zgadzała się
z tą, w której to miejsce było oglądane poprzednio (z uwzględnieniem ewentualnego
innego punktu obserwacji sceny). Tworzone przy użyciu metod SLAM mapy mogą być zarówno
dwuwymiarowe, jak i w pełni trójwymiarowe~\cite{6dslam}, przy czym w tym drugim przypadku
korzystając z odpowiednich sensorów możliwe jest tworzenie w pełni trójwymiarowych
modeli otoczenia, łącznie z informacją o kolorach~\cite{kinect_slam}.
Istnieje wiele gotowych implementacji algorytmów SLAM, z których wiele udostępnione
jest do dowolnego wykorzystania w internecie~\cite{openslam}.

\section{Generowanie trajektorii}

Ostatnim elementem składowym nawigacji robotów mobilnych jest generowanie trajektorii
do wykonania, a więc ścieżki z zadanego punktu początkowego (najczęściej aktualnej
pozycji robota) do ustalonego celu z ominięciem występujących po drodze przeszkód.
W szczeglnych przypadkach ścieżka ta może nie istnieć (np. z powodu otoczenia robota
przeszkodami), co również musi być sygnalizowane przez algorytmy planowania.
Przy generowaniu ścieżki wykorzystywane mogą być wszystkie informacje opisane wcześniej,
a więc mapa otoczenia, bieżące odczyty z czujników oraz wiedza o aktualnym położeniu
(nie jest jednak konieczne korzystanie ze wszystkich z nich, opisane dalej metody
używają wybranych podzbiorów).

\subsection{Podział metod generowania trajektorii}

Metody generowania trajektorii można podzielić ze względu na dostępność informacji
o środowisku, w którym robot będzie się poruszał (tutaj wyróżniamy metody lokalne
i globalne), ze względu na zupełność algorytmu (metody dokładne, przybliżone, losowe
oraz heurystyczne), a także ze względu na sposób reprezentacji informacji o środowisku
(tutaj, podobnie jak w mapach, wyróżniamy metody topologiczne i metryczne).

\subsubsection{Metody globalne i lokalne}

Ścieżki wyliczane przez algorytmy globalne opisują pełną drogę robota od punktu początkowego
do docelowego. Działanie tych algorytmów opiera się najczęściej na minimalizacji
funkcji kosztu (określanej w różny sposób), wykorzystując dostarczoną mapę środowiska
oraz określony model robota (opisujący zarówno jego właściwości fizyczne jak i dynamiczne).
Algorytmy te ze względu na swoją naturę i iteracyjny charakter działają najczęściej
w trybie off-line, czyli ścieżka jest przesyłana do robota dopiero po jej całkowitym
zaplanowaniu i dopiero wtedy robot rozpoczyna jej wykonywanie. Zaletą tych metod
jest teoretyczna optymalność znalezionych ścieżek, wadą jest konieczność powtórnego generowania
praktycznie całej ściezki, jeśli zmienią sie warunki otoczenia (podczas ruchu robota
dojdzie do sytuacji, w której nie będzie możliwe kontynuowanie wyznaczonej trajektorii).

Metody lokalne działają najczęściej w trybie on-line, a więc na bieżąco wprowadzają korekty
do ścieżki robota w miarę jak czujniki wykrywaja kolejne przeszkody. Metody te biorą
pod uwagę wszystkie przeszkody w otoczeniu robota (o ile są one wykrywane przez czujniki).
Mogą być stosowane jako jedyna metoda planowania (bez generowania globalnych trajektorii),
wtedy robot porusza się zawsze w kierunku punktu docelowego, a trajektoria jest wyznaczana
tak, aby nie wpaść na żadną przeszkodę. Wadą takiego podejścia jest często generowanie
ścieżek wielokrotnie dłuższych od optymalnej (np. próba dojazdu do innego pomieszczenia
może skończyć się najpierw przejazdem przez całe obecne pomieszczenie aż do ściany, a następnie
jazdy wzdłuż ścian i innych przeszkód aż do osiągnięcia celu). Z tego powodu często
stosuje się podejście mieszane -- zgrubna trajektoria planowana jest offline, a następnie
algorytmy lokalne mają podawane kolejne punkty na ścieżce globalnej i starają się do nich
dążyć (a więc śledzić ścieżkę globalną z uwzględnieniem nowych przeszkód).

\subsubsection{Zupełność algorytmu sterowania}

Algorytmy deterministyczne (dokładne) przeszukują wszystkie możliwe trajektorie,
a jeśli tylko ściezka do celu istnieje, to zostanie ona znaleziona. Bardzo często algorytmy
te są czasochłonne i mogą być stosowane jedynie przy przeszukiwaniu map topologicznych
(przeszukiwanie grafów). Podejścia przybliżone zakładają pewną dyskretyzację przestrzeni
poszukiwać (np. dyskretyzacja mapy czy ograniczenie możliwych sterowań robota do określonego
zbioru), a w tak ograniczonej przestrzeni są zupełne (a więc znalezienie rozwiązania
zależy od dobrego doboru parametrów dyskretyzacji).

Wyróżnić można także algorytmy losowe, a więc takie, które przeszukują przestrzeń
mozliwych rozwiązań w sposób niedeterministyczny, jednak przy odpowiednio długim czasie
(dążącym do nieszkończoności) prawdopodobieństwo znalezienia rozwiązania dąży do jedności.
Ostatnią grupą są algorytmu heurystyczne, a więc niezupełne, które wyszukują ścieżkę
w sposób przybliżony, a w czasie działania ... \textbf{BLABLABLA}

\subsubsection{Reprezentacja środowiska}

Podobnie jak w mapach, także przy algorytmach generowania ścieżki możemy wyróżnić
te działające na reprezentacji metrycznej oraz topologicznej. Nie jest to jednak
bezpośrednio związane z mapą, jaka jest dostarczana do algorytmu, gdyż może być tak,
że z dostarczonej mapy geometrycznej tworzony jest graf, a sam algorytm działa faktycznie
już na nim, będąc w istocie algorytmem topologicznym.

\subsection{Metody globalne}

W tej części przedstawione zostaną po krótce istniejące metody generowania globalnej
śieżki z podziałem ze względu na ogólną ideę ich działania. Podział dokonywany
jest uwzględniając metody dekompozycji i dyskretyzacji przestrzeni poszukiwań (jak zostało
to już wspomniane wcześniej, w ogólności algorytm nie musi działac na takiej samej reprezentacji
jak dostarczona mapa). Wyróżnić tutaj możemy trzy główne metody -- generowanie
,,map drogowych'' na dostępnej, wolnej przestrzeni, podział na komórki wolne i zajęte
a następnie tworzenie grafu sąsiedztwa, oraz metody wykorzystujące sztuczne pole potencjału.

\subsubsection{Mapy drogowe}

W metodach tych w pierwszej fazie tworzona jest sieć ścieżek łączących poszczególne
węzły (są one wybierane w różny sposób), a następnie do owej sieci dołączane są
także punkty początkowy i docelowy. Globalna trajektoria jest wybierana jako ścieżka
w grafie łącząca określone punkty. Jednym z takich algorytmów jest tworzenie grafów
widoczności, gdzie punktami charakterystycznymi są wierzchołki figur opisujących
przeszkody na mapie, a łączenie nastepuje wtedy, kiedy pomiędzy dwoma punktami można
poprowadzić prostą bez przecinania innych przeszkód. Ścieżki wygenerowane w ten sposób
mają jednak pewną wadę -- są niejako ,,przyklejone'' do przeszkód, przez co często
konieczne może okazać się korygowanie trajektorii w czasie jej wykonania.

Inaczej jest przy wykorzystaniu generowania mapy dróg z użyciem diagramów Voronoi -- w tym
przypadku segmenty drogi są umieszczone tak, aby być możliwie daleko od otaczających
przeszkód. Najpierw w każdym punkcie wolnej przestrzeni wyliczane są odległości
do najbliższej przeszkody, tworząc w punktach równo odległych od sąsiednich obiektów
grzbiet. Grzbiety te są następnie zamieniane w segmenty dróg, a ich przecięcia tworzą
węzły. Punkty startowy i końcowy są łączone krawędzią z najbliższymi węzłami i wyszukiwana
jest ścieżka pomiędzy nimi w tak utworzonym grafie. Metoda ta generuje ścieżki
dużo bezpieczniejsze, niż przy grafach widoczności.

\subsubsection{Podział przestrzeni}

Mając daną mapę w postaci geometrycznej możemy dokonać jej podziału w pewien ustalony
sposób (opisane w części dot. typów map). Przy dokładnym podziale na segmenty
odpowiadające obszarom wolnym i zajętym tworzony jest graf sąsiedztwa obszarów dostępnych
dla robota, a trajektoria jest wyznaczana jako ścieżka w tym grafie łącząca obszar
zawierający punkt startowy z obszarem zawierającym punkt docelowy. Konkretna trajektoria
pomiędzy sąsiednimi komórkami może być wyznaczona np. jako droga łącząca ich środki ciężkości.

Jeśli dyskretyzacja mapy nie jest oparta o cechy i kształt przeszkód, ale polega na
prostym podziale na regularną siatkę, to graf sąsiedztwa jest duż bardziej skomplikowany
i większy niż przy podziale dokładnym. W takim wypadu oczywiście można wyszukiwać
ścieżki bezpośrednio w tym grafie, są jednak metody robiące to w nieco inny sposób.
Polegają one na propagacji fali. Polegają one na systematycznym poruszaniu się
od komórki startowej, wpisując w sąsiednich komórkach koszt o jeden większy niż w obecnej.
Po wypełnieniu całej mapy (bądź szybciej, po wypełnieniu komórki docelowej) ścieżka
może być wyznaczona wstecz, jako łącząca komórki o coraz niższym koszcie.

\subsubsection{Metody sztucznego pola potencjału}

Metody te traktują robota jako sztuczny ładunek elektryczny umieszczony w polu potencjału
generowanego przez punkt docelowy o ładunku przeciwnym do ładunku robota
(przyciągający, a więc będący minimum funkcji) oraz przeszkody o ładunku zgodnym
z ładunkiem robota (a więc odpychające go, tworzące lokalne zbocza na funkcji).
Trajektoria generowana jest jako ściezka śledząca gradient funkcji w danym punkcie
i poruszająca się w kierunku największego spadku (a więc w kierunku celu). Istnieją
też odmiany algorytmu wykorzystujące informację o orientacji robota i generujące
siłę odpychającą od przeszkód w zależności od niej -- jeśli robot porusza się równolegle
do nich, to ich wpływ jest dużo mniejszy niż wtedy, kiedy porusza się w ich kierunku
(dzięki temu generowane trajektorie są dużo krótsze).

Problemem obu metod jest możliwość
wjechania w lokalne minimum funkcji potencjału, bądź wręcz w punkt o potencjale zerowym,
który nie jest punktem docelowym (np. w przypadku przeszkód nie będących wielokątami
wypukłymi). Metodą wyjechania z takiej sytuacji może być losowe zaszumianie trajektorii
bądź (już w czasie wykonania trajektorii) czasowe przełączenie się na algorytm śledzenia
ścian. Metody sztucznego pola potencjału są efektywne obliczeniowo i łatwo można wprowadzać
do nich nowe przeszkody, także on-line w trakcie jazdy robota. Dzięki temu mogą służyć
zarówno do globalnego planowania trasy jak i później do lokalnego omijania przeszkód.

\subsection{Metody lokalne}

\subsubsection{Algorytmy typu Bug}

Algorytmy typu Bug są najprostszymi algorytmami lokalnej modyfikacji
trajektorii i omijania przeszkód, polegają na podążaniu wzdłuż ich konturu.
Istnieją kilka głównych wariantów tej metody -- Bug0, Bug1, Bug2 i TangentBug.
Pierwsze trzy wymagają jedynie czujników zderzeniowych, ostatnia sensorów odległości,
wszystkie wymagają także informacji o przybliżonym kierunku do celu i jego położeniu.
Zależnie od wersji, przeszkody sa otaczane całkowicie, nawet więcej niż raz (Bug1),
bądź tylko do momentu wykrycia wolnej drogi do celu (Bug0, Bug2). W przypadku TangentBug
dodatkowo robot modyfikuje ścieżkę już zbliżając się do ceu bez konieczności kontaktu
z nim. Algorytm Bug1 jest dodatkowo zupełny, pozostałe nie.

%Najprostsza wersja algorytmu (Bug0) porusza się w kierunku celu aż do wykrycia przeszkody.
%Następnie porusza się wzdłuż jej konturu aż do momentu, kiedy będzie można znów poruszać
%się w kierunku celu, wtedy przestaje śledzić kontur.
%Kolejna wersja algorytmu po wykryciu przeszkody okrąża ją całkowicie, a następnie
%kontynuuje ruch w kierunku celu od miejsca, które na obwodzie przeszkody leży od niego
%najbliżej. Powoduje to bardzo mocne wydłużanie ścieżek, jednak jeśli tylko cel jest osiągalny,
%to algorytm ten zawsze znajdzie do neigo drogę. Wersja Bug2 nie okrąża całej przeszkody,
%objazd kończy się w momencie dojechania do linii łączącej położenie startowe z docelowym
%i od tego miejsca kontynuowana jest jazda na azymut. Ścieżki są krótsze niż w pierwszej wersji,
%nadal jednak nie są optymalne, dodatkowo istnieją takie konfiguracje przeszkód, przy
%których algorytm Bug2 się zapętla i nie znajduje drogi wyjścia (pomimo jej istnienia).
%Algorytm TangentBug wymaga zastosowania czujników odległości w celu wykrycia zbliżania
%się do przeszkód. Robot porusza się w kierunku celu, a jeśli w zasiędu jego czujników
%znajdą sie przeszkody, to rozpoczyna się modyfikacja trajektorii -- robot porusza się
%wtedy w kierunku punktu, który minimalizuje koszt ścieżki. Koszt jest liczony jako odległość
%robota od wykrytego punktu zsumowana z odległością tego punktu od celu. Jeśli w czasie
%ruchu funkcja kosztu rośnie, to robot rozpoczyna śledzenie przeszkody. Śledzenie kończy
%się, kiedy koszt zaczyna maleć, wtedy robot ponownie porusza się wprost na cel.

\subsubsection{Wyznaczanie kierunku ruchu}

Algorytmy z grupy VFH tworzą dookoła robota małą mapę zajętości, akumulując kolejne
pomiary z sensorów, dzięki czemu działają lepiej niż te z grupy Bug z czujnikami
o niewielkim zasięgu i polu widzenia. Na podstawie niej tworzony jest histogram
zawierający prawdopodobieństwo pojawienia się przeszkody pod danym kątem przed robotem.
Po progowaniu identyfikowane są w nim wolne miejsca na tyle duże, aby robot mógł
nimi przejechać, a z nich wybierane jest to o najmniejszym koszcie (np. będące najbardziej
w kierunku celu). Poprawione metody algorytmu (VFH+) biorą dodatkowo pod uwagę
ograniczenia kinematyczne robota, np. możliwy promień skrętu, przy wyborze kierunku ruchu.

Inną metodą z tej grupy jest diagram bliskości, będący w pewnym sensie rozwinięciem VFH.
Spisuje się on dużo lepiej w przypadku środowisk z wieloma przeszkodami, bierze też
pod uwagę kształt robota.

\subsubsection{Wyznaczanie prędkości kątowej i liniowej}

Metody opisane dotychczas bezpośrednio wyznaczały kierunek, w którym powinien poruszać
się robot aby uniknąć zderzenia bądź ominąć wykrytą przeszkodę. Innym podejściem
jest generowanie zbioru możliwych sterowań (a więc zadanej prędkości liniowej i kątowej)
i sprawdzanie, które z nich są możliwe do wykonania w danym momencie, a które prowadzą
do zderzenia z przeszkodami. W przestrzeni konfiguracyjnej robota wprowadza się ograniczenia
zarówno na prędkości robota, jak i na przyspieszenia. Metoda krzywizn i prędkości
(Curvature velocity, CVM) przybliża kształt przeszkód okręgami (ze względu na wydajność
obliczeń), a następnie wyznacza takie sterowania, aby w kroku o zadanej długości
nie doprowadzić do zderzenia, jednocześnie maksymalizując prędkość liniową i poruszając
się w kierunku celu.

Metoda okna dynamicznego (Dynamic Window Approach, DWA) tworzy wokół robota w przestrzeni
konfiguracyjnej okno zawierające możliwe w danym momencie sterowania, a następnie nakłada
je na bieżącą mapę zajętości. Pozostają jedynie te sterowania, które zapewniają,
że robot zdąży zatrzymać się przed przeszkodą. Z pozostałych sterowań wybierane jest
to o najniższym koszcie (podobnie jak w CVM -- możliwie duża prędkość liniowa,
zachowanie odległości od przeszód i poruszanie się w kierunku celu).
