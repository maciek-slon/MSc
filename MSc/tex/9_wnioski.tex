% !TeX root = main.tex


\begin{savequote}[70mm]
,,Nie płacz, że coś się skończyło, tylko uśmiechaj się, że ci się to przytrafiło.''
\qauthor{Gabriel García Márquez}
\end{savequote}


\chapter{Wnioski}
\label{chap:wnioski}

\section{Rezultat pracy w obliczu założeń}

Stworzony system sterowania robota mobilnego spełnia założenia określone na początku pracy.
Dzięki modularnej budowie i~pełnemu odseparowaniu poszczególnych jego elementów
zgodnie z~ich logicznymi zależnościami uzyskano wysoką elastyczność i~umożliwiono
łatwą wymianę pojedynczych jego fragmentów.

Wyniki uzyskane w przeprowadzonych testach pokazują, że informacja z~czujników otoczenia
działających w~trzech wymiarach może być z~powodzeniem wykorzystana podczas nawigacji
robotów mobilnych, zarówno w~trybie symulowanego skanera laserowego, jak i~wykorzystując
w~pełni uzyskiwaną chmurę punktów. Dodatkowe obciążenie wprowadzane przez przetwarzanie
większej ilości danych jest kompensowane przez dużo szersze spektrum możliwych do
wykrycia przeszkód, zarówno tych stosunkowo mniej niebezpiecznych, jak małe przedmioty,
jak i~mogących spowodować duże uszkodzenia robota, np. schody prowadzące w~dół.


\section{Perspektywy rozwoju}

Sama struktura systemu sterowania stworzonego w~ramach pracy jest rozwiązaniem kompletnym,
perspektywy rozwoju dotyczą raczej rozbudowy poszczególnych jego elementów. Po pierwsze w~najbliższym
czasie zmianie ulegną sprzętowe sterowniki robota Elektron, w~związku z~czym trzeba będzie
przygotować nowe moduły komunikacyjne. Innym kierunkiem rozwoju jest tworzenie nowych
aplikacji wykorzystujących stworzone już elementy, jak na przykład śledzenie poruszających
się obiektów czy połączenie aplikacji eksplorujących nieznany teren z~zadaniami tworzenia
map.

W~dynamicznie zmiennym otoczeniu, jakim bez wątpienia są pomieszczenia biurowe oraz
przestrzenie laboratoryjne, sensory trójwymiarowe są praktycznie jedyną możliwą do
zastosowania formą niezawodnego i~szybkiego wykrywania przeszkód. Stworzony system
można wykorzystać na mobilnych robotach usługowych działających w~takim środowisku
bez konieczności przystosowywania otoczenia do pracy robota. W~szczególności nie jest
wymagane ani usuwanie przeszkód, ani specjalne oznaczanie przestrzeni dostępnych dla
robota (poprzez np. oznaczenia na podłodze), dzięki czemu robot może być w~krótkim
czasie uruchomiony w~nowym środowisku, a~podczas jego działania nie jest konieczne
opuszczanie tych pomieszczeń przez ludzi. Do zadań, które mogą być realizowane przez
mobilne roboty usługowe należą chociażby pomoc w~laboratoriach badawczych (różnego
rodzaju ruchome stoły narzędziowe), platformy informacyjno-usługowe na lotniskach
czy roboty telekonferencyjne w~biurach.
