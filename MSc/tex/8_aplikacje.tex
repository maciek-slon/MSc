% !TeX root = main.tex


\begin{savequote}[70mm]
,,''
\qauthor{}
\end{savequote}


\chapter{Aplikacje}
\label{chap:aplikacje}

\section{Automatyczna kalibracja odometrii i żyroskopu}

Proces kalibracji odometrii oraz pomiarów uzyskiwanych z żyroskopu został
przestawiony w rozdziale~\ref{chap:software}. W celu jego ułatwienia i
zautomatyzowania stworzona została aplikacja przeprowadzająca dwuetapową
kalibrację -- najpierw wyliczająca poprawki dla współczynników kątowych
żyroskopu i odometrii, a następnie licząca poprawkę dla składnika liniowego
odometrii. Aby poprawnie przeprowadzić cały proces wymagana jest pusta
przestrzeń o powierzchni ok. 2x2m, oraz płaska powierzchnia na jednym z końców
(np. ściana).

\subsection{Wyznaczanie odległości i kąta obrotu względem ściany}

W celu wyznaczenia faktycznej wartości obrotu wykonanego przez robota oraz
określenia pokonanej odległości, wymagana jest obecność sensora zwracającego
odczyty laserowe (w przypadku robota Elektron możliwe jest wykorzystanie
zarówno skanera SICK jak i sensora Kinect z dodatkowym komponentem
zamieniającym pomiary z chmury punktów na symulację odczytu laserowego). Na
podstawie punktów zwracanych przez sensor ustawiony na przeciwko ściany
wyliczana jest orientacja robota względem niej, a także odległość.

\subsubsection{Orientacja}

Dane odczytane z lasera mają postać zbioru punktów we współrzędnych biegunowych.
Pierwszym krokiem jest ich przeliczenie na współrzędne kartezjańskie w układzie
związanym ze środkiem sensora. Po takim przeliczeniu, wykorzystując metodę
najmniejszych kwadratów, wyliczane są parametry prostej przechodzącej możliwie
blisko danych punktów. Współczynnik kierunkowy tej prostej określa jednocześnie
orientację robota względem ściany.

Mając dane punkty w postaci $(x_i, y_i)$, gdzie $i=1\ldots n$, oraz przyjmując
oznaczenia:

\[
S_x = \sum_{i=1}^n x_i \mathsp
S_y = \sum_{i=1}^n y_i \mathsp
S_{xx} = \sum_{i=1}^n x_i^2 \mathsp
S_{xy} = \sum_{i=1}^n x_i \cdot y_i \mathsp
%S_{yy} = \sum_{i=1}^n y_i^2 \mathsp
\Delta = n \cdot S_{xx} - S_x^2
\]

Współczynniki prostej o równaniu $y=a\cdot x+b$ wyznaczane są przez wzory:

\[
a = \frac{n \cdot S_{xy} - S_x \cdot S_y}{\Delta} \mathsp
b = \frac{S_{xx} \cdot S_y - S_x \cdot S_{xy}}{\Delta}
\]

Skąd orientację robota względem ściany (wyznaczonej prostej) uzyskujemy ze
wzoru:

\[
\phi=atan(a)
\]

\subsubsection{Odległość}

Wykorzystując już punkty we współrzędnych kartezjańskich, odległość od ściany
wyliczana jest jako średnia z wartości $y$ wszystkich punktów. 

\subsection{Właściwy proces kalibracji}

Przed rozpoczęciem kalibracji robot powinien zostać ustawiony na wprost ściany,
w odległości ok. 50cm od niej. Pierwszą czynnością po uruchomieniu zadania jest
wyznaczenie składowej stałej pomiarów z żyroskopu, po czym następują trzy
pełne obroty robota z różnymi prędkościami (fakt wykonania pełnego obrotu
wyznaczany jest na podstawie odczytów z odometrii). Po wykonaniu każdego obrotu
wyliczana jest faktyczna jego wartość (na podstawie porównania orientacji
względem ściany zmierzonej przed i po wykonaniu obrotu), a razem z nią
zapamiętywane są wartości obrotu odczytane z odometrii i żyroskopu. Pomiędzy
kolejnymi pomiarami robot jest pozycjonowany powtórnie na wprost ściany (z
pewną, niewielką tolerancją) wykorzystując pomiary laserowe. Po wykonaniu
wszystkich obrotów wyliczane są błędy wskazań odometrii i żyroskopu (stosunek
wartości odczytanej do zmierzonej laserem), a z nich wyliczana jest średnia
wartość poprawki, którą należy wprowadzić do współczynników obrotowych obu
badanych czujników.

Po wykonaniu kalibracji obrotów robot powtórnie ustawia się na wprost ściany, a
następnie odjeżdża od niej na odległość ok. 2 metrów. Następnie przejeżdża 1.5
metra do przodu, a po przejechaniu tego dystansu wyliczana jest poprawka
współczynnika liniowego odometrii (stosunek przejechanej odległości wyliczonej
przez odometrię do odległości uzyskanej z pomiarów laserowych). Ostateczna
wartość poprawki wyliczana jest jako średnia z trzech takich przejazdów z
różnymi prędkościami. Uzyskane w tym procesie wartości współczynników należy
pomnożyć przez współczynniki już ustawione w sterownikach robota.



\section{Prosta, losowa eksploracja}

\section{Budowa mapy}

\section{Dojazd do wyznaczonego celu z omijaniem przeszkód}
