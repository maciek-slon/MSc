% !TeX root = main.tex


\begin{savequote}[90mm]
\begin{enumerate}
\item Robot nie może skrzywdzić człowieka, ani przez zaniechanie działania dopuścić, aby człowiek doznał krzywdy.\\
\item Robot musi być posłuszny rozkazom człowieka, chyba że stoją one w~sprzeczności z~Pierwszym Prawem.\\
\item Robot musi chronić sam siebie, jeśli tylko nie stoi to w~sprzeczności z~Pierwszym lub Drugim Prawem.\end{enumerate}
\qauthor{Isaac Asimov}
\end{savequote}


\chapter{Wstęp}
\label{chap:wstep}

\section{Motywacja}

Nawigacja autonomicznego bądź semiautonomicznego robota mobilnego wymaga możliwie
efektywnych i~dokładnych metod analizy otoczenia oraz wykrywania przeszkód.
Najczęściej stosuje się w~tym celu całe zestawy różnych czujników, z~których część
podaje odczyty bądź punktowe (np. czujniki zderzeniowe bądź sensory odległości
wykorzystujące podczerwień albo ultradźwięki) bądź czasami dwuwymiarowe (skanery
laserowe podające odległość do otaczających przedmiotów, jednak odczyty wykonywane
jedynie w~jednej płaszczyźnie). Taka konfiguracja czujników umożliwia dość sprawne
poruszanie się w~nieznanym środowisku, jednak istnieje szereg przeszkód, których za
ich pomocą nie da się wykryć. Należą do nich chociażby małe obiekty znajdujące
się bardzo blisko ziemi (czujniki najczęściej są umieszczone na wysokości od kilku
do kilkunastu centymetrów) czy obiekty zwisające z~góry mogące zahaczyć o~wystające
części przejeżdżającego pod nimi robota.

Wykorzystanie informacji dostarczanych przez czujniki 3D (a więc zwracające informacje
o~głębi w~każdym punkcie obserwowanego obrazu w~przypadku czujników opartych o~kamery)
pozwala na wykrywanie dużo szerszego spektrum obiektów, a~więc sprawniejsze i~bezpieczniejsze
poruszanie się do określonego celu.

\section{Cel pracy}

Celem pracy było stworzenie pełnego systemu nawigacji robota mobilnego,
wykorzystującego informacje o~otoczeniu pochodzące z~wielu źródeł, w~tym z~kamery 3D.
Należało także dokonać wyboru konkretnej metody obrazowania trójwymiarowego
spośród dostępnych rozwiązań z~uwzględnieniem warunków, w~których będzie ono
wykorzystywane i~ograniczeń sprzętowych platformy.

Dodatkowo przygotowany system nawigacji miał pozwalać na łatwe modyfikowanie
i~zmianę poszczególnych modułów (m.in. planerów trasy), a~więc powinien być
gotowym stanowiskiem badawczym i~eksperymentalnym. Jego przeznaczeniem miało
być badanie różnych algorytmów sterowania, budowy map, generacji trajektorii czy lokalizacji.

%Stanowisko do badania algorytmów sterowania, budowania map, nawigacji i~inne takie
%propagandowe teksty, ogólnie chodziło o~przygotowanie niezawodnej platformy
%gotowej na prostą podmianę pojedynczych algorytmów bez potrzeby głębokich
%zmian w~pozostałych elementach.

\section{Struktura pracy}

Na początku pracy (rozdział~\ref{chap:porownanie}) skupiono się na porównaniu
dostępnych metod pozyskiwania trójwymiarowego
obrazu sceny. Wyniki przeprowadzonych doświadczeń pozwoliły wybrać docelowe rozwiązanie
wykorzystane przy realizacji zadania. Po nim następuje
teoretyczne wprowadzenie w~podstawy nawigacji robotów mobilnych (rozdział~\ref{chap:nawigacja})
zawierający opis najważniejszych konceptów z~dziedziny wraz z~przeglądem istniejących
rozwiązań. Rozdział~\ref{chap:sprzet} przedstawia wykorzystaną platformę sprzętową,
a~więc samego robota mobilnego Elektron, jak i~pozostałe moduły sensoryczne. Kolejne
dwie części dotyczą części programowej całego systemu -- rozdział~\ref{chap:programowanie}
zawiera przegląd metod programowania robotów mobilnych oraz wybór zastosowanego rozwiązania,
a~w~rozdziale~\ref{chap:software} pokazana jest warstwa pośrednicząca odbierająca
dane z~czujników i~przekształcająca je do postaci bezpośrednio użytecznej w~sterowaniu.
W rozdziale~\ref{chap:mapa} omówiony został w~pełni system nawigacji przygotowany
na omawianej platformie, wraz z~opisem implementacji kluczowych elementów
przetwarzania i~filtrowania danych sensorycznych.
Rozdział~\ref{chap:aplikacje} przedstawia przekrojowo aplikacje przygotowane
przy użyciu zaproponowanych technik mapowania i~nawigacji, zaczynając od zadań
najprostszych, takich jak losowa eksploracja terenu, kończąc na złożonych zadaniach
zawierających moduły mapowania, samolokalizacji i~omijania przeszkód. Rozdział ten
zawiera również opis procedury testowania przygotowanych algorytmów. Praca zakończona
jest podsumowaniem przeprowadzonych badań i~doświadczeń.


